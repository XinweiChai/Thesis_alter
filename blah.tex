Les systèmes concurrents présentent un intérêt depuis des décennies. Avec leur sémantique simple mais expressive, les systèmes concurrents deviennent un bon choix pour ajuster les données et analyser les mécanismes sous-jacents. Cependant, l'apprentissage et l'analyse de tels systèmes concurrents sont difficiles pour ce qui concerne les calculs. Lorsqu'il s'agit de grands ensembles de données, les techniques les plus récentes semblent insuffisantes, que ce soit en termes d'efficacité ou de précision.

Dans cette thèse, nous proposons un cadre de modélisation raffiné ABAN (Asynchronous Binary Automata Network) et développons des techniques d'analyse d'atteignabilité basées sur ABAN: PermReach (Reachability via Permutation search) et ASPReach (Reachability via Answer Set Programming). Nous proposons ensuite deux méthodes de construction et d'appren-tissage des modèles: CRAC (Completion via Reachability And Correlations) et M2RIT (Model Revision via Reachability and Interpretation Transitions) en utilisant respectivement des données continues et discrètes pour s'ajuster au modèle et des propriétés d'accessibilité afin de contraindre les modèles résultants.

Le chapitre~\ref{chap:intro} décrit brièvement le contexte et la contribution de nos recherches. Le chapitre~\ref{chap:stateOfTheArt} présente l'état de l'art des modélisations, des model checkers, des différentes dynamiques associé aux modès et les techniques d'apprentissage des modèles. Certains d'entre eux sont référencés dans les chapitres suivants.

Le chapitre~\ref{chap:refinement} présente notre cadre de modélisation et ses analyseurs d'accessibilité associés, qui sont basés sur l'analyse statique. Nous nous concentrons sur les cas non concluants d'analyse statique pure et extrayons les composants clés empêchant une solution directe. Nous appliquons ensuite des heuristiques sur ces composants, en les résolvant avec une recherche limitée pour obtenir un résultat plus concluant du problème d'accessibilité.

Le chapitre~\ref{chap:modelInference} présente la méthodologie de l'apprentissage par modèle. Nos systèmes de construction de modèles par apprentissage CRAC et M2RIT effectuent en fait une sélection des modèles. Ils choisissent un modèle parmi les candidats qui satisfont à toutes les contraintes d'accessibilité données. Cependant, le nombre de modèles candidats pouvant être de très grande taille, nos réviseurs de modèles peuvent réduire l'espace de recherche avec des contraintes lors de la génération des modèles.

Le chapitre~\ref{chap:test} présente quelques tests comparatifs et exploratoires et leurs résultats sur les méthodes présentées aux chapitres~\ref{chap:refinement} et \ref{chap:modelInference}. PermReach et ASPReach sont plus efficaces que les vérificateurs de modèle traditionnels pour l'analyse de l'accessibilité. Ils effectuent une analyse plus concluante tout en maintenant la durée de fonctionnement à la même échelle que les analyseurs statiques purs.

Le chapitre~\ref{chap:conclusion} conclut la thèse et propose des travaux futurs possibles.